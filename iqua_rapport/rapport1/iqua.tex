\documentclass{report}
\usepackage[utf8]{inputenc}
\usepackage[francais]{babel}
\usepackage{hyperref}
\usepackage{url}
\usepackage{multirow}
\usepackage{shorttoc}
\author{bog}

\title{Projet Iqua : Observations et propositions}

\begin{document}
\maketitle{}

\tableofcontents{}
%\shorttoc{Sommaire}{3}

\chapter*{Remerciements}

\paragraph{}
Je tiens particulièrement à remercier \textsc{Slackbot} pour son dévouement et pour ses réponses et conseils pertinant sur Slack.

\newpage
\chapter{Préambule}
\paragraph{}
Voici un petit rapport personnel concernant le projet Iqua. Dans ce document, je vais faire état de ma vision de l'équipe et du jeu. Je vais paraitre très critique, mais je suis conscient de la difficulté qu'on peut éprouver à travailler sur un tel projet et à vivre sa vie en même temps. Pour moi Iqua n'est qu'un projet : c'est-à-dire qu'il n'est jamais prioritaire face aux événements de la vie. On le réalise sur notre temps libre, et par plaisir. J'ai tout cela en tête.

\paragraph{}
De plus, les critiques que je serai amener à faire ne concerne jamais une personne en particulier, mais notre fonctionnement général. Quand je dénonce nos tendances, c'est pour pouvoir les améliorer, et non pas pour pointer du doigt une personne.

\paragraph{}
Enfin, je n'avais pas l'intention d'écrire un put*in de rapport de cette taille. C'est en commençant à écrire un paragraphe sur slack que j'ai réalisé du nombre de chose que je voulais vous dire. De plus, ce document n'est pas exhaustif : je l'écris avant tout pour avoir un support de réflexion et éventuellement de débat et pour suciter des réactions. Bref, en espérant vexer personne et avec l'espoir d'engendrer une amélioration de notre équipe.
\\ \\
\begin{center}
  \paragraph{}
  \textsc{Bonne lecture !}
\end{center}

\chapter{L'équipe}
\section{Mauvaise priorité}
\paragraph{}
Il semble que nous sommes tous plus ou moins indisponible pour iqua : il est par conséquent important\footnote{En effet, j'estime que notre indisponibilité est normal, et que nous devons seulement en être conscient et le gérer.} de bien réfléchir à la direction et à la forme que prendront nos futurs contributions au projet.
Je pense que nous sommes dans une période du processus créatif dans laquelle le plus important reste l'établissement des bases du jeu.
Autrement dit, il nous faut un prototype jouable et amusant avant tout, sans exception. 

\paragraph{}
Je me questionne sur l'intérêt de concevoir le level design de l'intégralité du jeu maintenant. Pourquoi se lancer dans une tâche aussi titanesque 
- qui de tout évidence ne nous plaira probablement plus au moment de l'inclure dans le jeu du fait de l'évolution futur de ce dernier - au lieu de diriger notre énergie 
vers la conception d'un prototype efficace ? De plus, nous ne sommes pas capable de le faire. Nous n'avons pas l'expérience de notre propre jeu (qui viendra lors des 
futurs réflexions au sujet du game design à l'aide du prototype), et nous n'avons pas le temps.

\paragraph{}
Notre jeu est un jeu à contenu. Il y aura un moment pour produire du contenu en masse ; ce moment n'est à mon sens pas encore arrivé. 
Je me question d'ailleurs sur mon rôle dans l'équipe lorsque arrivera cette phase (au-delà de la maintenance du code).

\paragraph{}
Je pense donc qu'il est nécessaire pour nous de faire un prototype sérieux, d'écrire des outils de création de contenu, et de produire ce dernier.
Si nous ne suivons pas ce chemin, nous nous égarerons totalement. Nous traînerons un demi-jeu inachevé, au contenu incomplet, bugguer, amplis d'incohérence, mal accordé,
dissonant, désorganisé et décevant.

\paragraph{}
Sans méthode, nous nous perdrons et iqua sera écrasé sous une masse de travail, et la démotivation se chargera de le ballayer du revers de la main.

\section{Absence de plan}
\paragraph{}

\paragraph{}
Nous n'avons pas de plan si ce n'est les objectifs que nous nous fixons lors des réunions au Cub'. 
Autrement dis, notre seul moteur sont nos rencontres avec J-C Brocard (que nous traitons comme "\textit{notre bâton sans carotte}").
Je crains que cela soit trop insuffisant comme facteur d'avancement. Nous faisons un jeu très ambitieux relativement à notre expérience, et \textbf{cela à un prix}.
Ce prix va être le travail et la motivation. Ne pas croire que ça va marcher, mais le savoir et \textit{se donner les moyens} de l'accomplir.
Je ne dis pas que faire notre travail au dernier moment, et trainer est inacceptable. Je pense juste que cela nous nuis grandement.
Je ne refuse pas l'échec, je ne le déteste pas. On le vivra peut-être si on ne s'organise pas mieux (et ce n'est pas grave).
Mais le succès personnel derrière Iqua est, je trouve, très attirant.

\paragraph{}
Ainsi, je pense qu'il nous faut un vrai plan (sur le long terme). Certains détails n'ont pas été abordé, comme par exemple le temps que nous nous donnons pour terminer le jeu. Il n'est pas souhaitable de laisser le développement trainer tout comme il est important d'avoir un temps suffisant pour avoir un jeu de qualité. Il est nécessaire d'en parler. Nous discutons trop individuellement, et nous manquons de communication collective.

\paragraph{}
Nous nous donnons des tâches immenses à réaliser dans des temps courts, sans même savoir où nous allons !
Nous courrons vers un mur à grand pas en ne tenant compte que du prochain pas que nous ferrons et sans nous préoccuper du mur.

\paragraph{}
\label{independanceCub}
Donnons nous une date et une vision du jeu. Soyons un peu indépendant du Cub' pour déterminer la route que nous souhaitons emprunter.
Le Cub' n'est pas notre employeur, il s'agit d'une aide précieuse. Nous sommes aidé, pas assisté, ne l'oublions pas !

\section{L'équipe meurt}

\paragraph{}
Un titre provocant pour introduire une réalité qui va se réaliser si on ne fait pas attention. Nous avons un peu de temps, nous avons des compétences, nous avons le soutiens du Cub'. Mais tout cela est inutile sans une motivation réelle de notre part. Une motivation réelle et une implication toute aussi réelle. Je ne dis pas que nous ne sommes pas motivé. Seulement, notre motivation est basse. Pire encore, elle semble faiblir.

\paragraph{}
Voici un début de réflexion concernant les raisons de la faible motivation ambiante.

\begin{center}
  \begin{tabular}{c | c}
    %\hline
    \textsc{Problèmes} & \textsc{Solutions possibles} \\
    \hline
    
    \multirow{3}{*}{Manque de vision du jeu} &  Nommer un responsable   \\
    & pour la rédaction d'un \\
    & Game Design Document.\\
    
    \hline
    
    \multirow{5}{*}{Manque d'immersion dans le développement} &  Établir un planning des disponibilités\\
    & des membres de l'équipe afin de\\
    & pouvoir organiser des réunions.\\
    & \\
    & S'obliger à plus communiquer sur Slack.\\

    \hline
    Ampleur de la tâche & Réduire le contenu du jeu.\\
    \hline
    \multirow{2}{*}{Manque de résultats visibles} & Créer des outils afin d'inclure\\
    & le contenu facilement dans le jeu.\\
    %\hline
  \end{tabular}
\end{center}

\paragraph{}
Ainsi, \textbf{la motivation n'est pas un élément à négliger} sous peine de voir tout le projet s'effondrer de l'intérieur. Je peux sembler alarmiste, mais je vous assure que c'est réellement \textit{déprimant\footnote{Souvenez vous, ancien membre de la PeruTeam, que notre seule consolation dans le jeu de la game jam fut le sentiment d'avoir fait au mieux et de ne pas avoir abandonné.} et culpabilisant} de se voir abandonner un projet du fait d'un manque de motivation. Je pense que face à une baisse de motivation qui s'annonce, le mieux est d'arrêter l'aventure. C'est pourquoi je pense que si un membre\footnote{Je ne pense pas que quelqu'un désire quitter le groupe à l'heure actuelle, mais c'est une situations qu'il faut envisager.} de l'équipe ne souhaite plus travailler sur Iqua, nous devrions le laisser partir sans discuter sa décision. Il est certain que le départ de tous signerait la mort d'Iqua, mais je ne pense pas que ça arrivera. Je pense également qu'il ne faudrait virer personne : je sais que parfois le manque d'implication peut poser problème (surtout si on a promit\footnote{cf - ma vision du rôle du Cub, page \pageref{independanceCub} } des choses), mais pas question de se comporter comme des pseudo-employeurs ayant ce pouvoir. Encore une fois, je parle de décisions extrèmes pour le groupe, mais je préfère donner mon avis maintenant au cas où ça pourrait influencer\footnote{Dans le bon sens et dans l'intérêt du groupe et du projet bien sûr ;-) } le déroulement de la gestion du groupe.

\chapter{Le jeu}

\section{Le prototype}
\paragraph{}
Je ne saurais plus insister sur l'importance de le développer avant tout. C'est lui qui va nous guider dans la conception du jeu, lui qui nous révélera les défauts à corriger et les élements à développer. Toute l'énergie donnée à Iqua devrait aller sur cette démo\footnote{``Démo'' ou ``Prototype'', peu importe le nom pourvu que le but reste.}.

\paragraph{}
Je ne reproche pas au groupe de ne pas développer le prototype (il suffit de regarder ce dernier pour voir qu'il évolue bien). Je pense en revanche que nous mettons beacuoup d'énergie dans des tâches prématurées (level design avancé, scénarisation avancée, génération de contenu avancé).

\paragraph{}
Afin de faire évoluer plus rapidement notre démo, je pense qu'il serait intéressant de maintenir une liste des corrections à apporter (changer tel élement de l'interface, modifier la largeur de tel bouton etc).

\section{La jouabilité}
\subsection{Narrativité}
\paragraph{}
Pour le moment, nous sommes en train de construire une histoire interactive où le joueur remplis des objectifs afin de faire progresser la narration.
Je pense que c'est insuffisant pour faire un bon jeu à moins que le scénario soit particulièrement intéressant. Je part aussi du principe qu'il le sera.

\paragraph{}
On se heurte alors à une difficulté monstre. En effet, les quêtes risquent de ralentir la narration. \textit{Tiens, tu ne voudrais pas aller chercher la poupée de ma petite soeur ?}, quête sympathique aux premiers abords, peut très rapidement devenir vecteur de coupure dans le jeu. La narration se retrouverait alors segmentée par des nombreuses quêtes . Je parle là de risque, pas de nécessité.

\paragraph{}
Du fait de sa forte tendance narrative, le jeu ne doit pas demander au joueur d'avoir de la dextérité ou même des compétences intellectuelles. En effet, le jeu étant basé sur l'exploration d'un monde au travers d'une histoire, il n'est pas possible\footnote{J'expose mon point de vue, je suis conscient que des jeux appréciés le font - mais généralement je n'aime pas ces jeux pour cette raison.} d'ajouter du challenge sans amenuiser la narration (voir la gâcher). C'est pourquoi je pense que le jeu doit se limiter à faire découvrir un monde, et à faire passer un bon moment d'évasion dans un imaginaire passionnant.

\paragraph{}
Se pose alors la question de la difficulté du jeu. Je vais faire vite : pour moi cette question serait hors-propos, tout comme serait hors-propos le fait de discuter de la difficulté d'un livre. On ne peut pas parler de difficulté dans un jeu narratif. Ou alors on se trompe.

\paragraph{}
Développer un jeu narrative est intéressant, car cela donne la possibilité de toucher au domaine de l'imaginaire du joueur. Cependant, il est nécessaire de bien intégrer la narration au sein du jeu (car elle en est le coeur), et de se méfier des élements de jouabilité qui risqueraient de faire la vaciller et de la ruiner.

\subsection{Élements additionnels}

\paragraph{}
Une narration, au sein d'un jeu narrative, est incontestablement obligatoire. Mais, sans élement additionnel, le jeu ne risquerait-il pas de devenir trop linéaire et ennuyeux ? Certes, porté par le scénario, il jouirait d'une longue durée de vie et d'une courbe d'intérêt parfaitement maîtrisée. Mais on peut vouloir plus.

\paragraph{}
Car n'oublions pas qu'en créant un jeu, nous sommes en fait en train de \textit{créer une expérience}. Nous contrôlons cette expérience, et nous faisons en sorte qu'elle soit le plus intéressante possible pour le joueur. Afin d'y parvenir, nous devons également donner la possibilité à ce dernier d'avoir une influence sur son expérience. Avec une narration solide, notre jeu s'adresse au joueur. Il est maintenant temps d'introduire des élements de jouabilité permettant au joueur de s'adresser au jeu. Ces élements de jouabilité ne doivent pas fournir au joueur un challenge au risque de ralentir la narration.

\paragraph{}
Bien entendu, nous devrons réfléchir à ces élements additionnels afin d'enrichir le jeu. N'oublions pas que ces élements peuvent trouver leurs formes dans le vocabulaire et l'imaginaire de la magie. Nous pouvons nous inspirer également des académies du jeu. Comment intégrer les académies avec le monde jeu ? Chaque académie porte la narration et introduit des personnages et des situations. Ce rôle là est clair et se montrera fort probablement efficace. Mais quant n'est-il des cours et de la vie étudiante en ces lieux magiques ? Quel rôle aura le joueur au sein des académies ?

\subsection{Personnification du joueur}
\paragraph{}
Qui est le joueur au sein du jeu ? Visiblement il parle (via une interface de dialogue) et peut posséder des objets (via une interface d'inventaire). Je dois en conclure qu'il est donc un personnage du monde, vu qu'il agit dessus. Mais en tant que personnage du monde, ne possède t-il pas de nom ? N'a t-il aucune apparence ? N'a t-il aucun but ?
Du fait de la vue à la première personne, il est normal qu'on ne puisse le voir, mais pourquoi est t-il si peu important dans les mécaniques de jeu ? Il semble n'être qu'une enveloppe vide possédée par l'esprit du joueur pour servir le monde du jeu.

\paragraph{}
Il serait intéressant de permettre au joueur de personnaliser cette enveloppe vie (ne serait-ce qu'ne la nommant). De plus, le joueur doit avoir un but au delà des quêtes. Le jeu étant narratif, ce but doit justifier le voyage du joueur au sein du monde du jeu, et son regard candide sur celui-ci.



\subsection{De la rejouabilité}
\paragraph{}
Une histoire linéaire, un nombre de quête fini, pas (encore) d'élement de jouabilité à côté de la narration. À vue de nez, je dirais que le jeu aura une rejouabilité nulle.
Il peut être intéressant de concevoir les élements additionnels avec ce constat en tête. En introduisant de la rejouabilité, nous permettons au joueur de profiter pleinement de l'expérience délivrée par le jeu. Afin d'améliorer cette facette du jeu, je pense qu'il serait intéressant de concevoir les élements de rejouabilité d'une certaine manière\footnote{Que j'avais commencé à décrire sur slack mais qui a relativement déplu.}.

\paragraph{Le procédurale}\footnote{\og En informatique, la génération procédurale est le fait de créer du contenu à la volée. Cette idée est souvent liée aux applications de synthèse d'image et au level design dans les jeux vidéo. \fg \hspace{0.5cm} Wikipédia, août 2015}
est un premier moyen de générer une part du contenu, qui de part sa nature changera à chaque partie que le joueur commencera.

\paragraph{Le combinatoire}
 permettrait au joueur ou au jeu de combiner des élements pré-existant afin d'un créer des nouveaux. Ainsi, le joueur aura la possibilité de faire des ``mélanges'' et de joueur avec certains élements du jeu en les combinants.

\chapter{Conclusion}
\paragraph{}
Voila, je n'ai plus grand chose qui me vient à l'esprit, c'est pourquoi j'achève ce rapport là. Pourquoi s'être donné autant de mal pour écrire ce document ? Et bien j'ai espoir qu'il permette d'améliorer un peu notre équipe en fournissant des pistes de réflexions et en étant un support de discussion.

\paragraph{}
J'espère qu'on trouvera le temps de discuter de tout ça sur Slack !

\vspace{3cm}

\begin{center}
\paragraph{}
Et comme dirait Slackbot : Vive le projet iqua !
\end{center}
  
\end{document}
